% Options for packages loaded elsewhere
\PassOptionsToPackage{unicode}{hyperref}
\PassOptionsToPackage{hyphens}{url}
%
\documentclass[
]{article}
\title{taller 2}
\author{}
\date{\vspace{-2.5em}}

\usepackage{amsmath,amssymb}
\usepackage{lmodern}
\usepackage{iftex}
\ifPDFTeX
  \usepackage[T1]{fontenc}
  \usepackage[utf8]{inputenc}
  \usepackage{textcomp} % provide euro and other symbols
\else % if luatex or xetex
  \usepackage{unicode-math}
  \defaultfontfeatures{Scale=MatchLowercase}
  \defaultfontfeatures[\rmfamily]{Ligatures=TeX,Scale=1}
\fi
% Use upquote if available, for straight quotes in verbatim environments
\IfFileExists{upquote.sty}{\usepackage{upquote}}{}
\IfFileExists{microtype.sty}{% use microtype if available
  \usepackage[]{microtype}
  \UseMicrotypeSet[protrusion]{basicmath} % disable protrusion for tt fonts
}{}
\makeatletter
\@ifundefined{KOMAClassName}{% if non-KOMA class
  \IfFileExists{parskip.sty}{%
    \usepackage{parskip}
  }{% else
    \setlength{\parindent}{0pt}
    \setlength{\parskip}{6pt plus 2pt minus 1pt}}
}{% if KOMA class
  \KOMAoptions{parskip=half}}
\makeatother
\usepackage{xcolor}
\IfFileExists{xurl.sty}{\usepackage{xurl}}{} % add URL line breaks if available
\IfFileExists{bookmark.sty}{\usepackage{bookmark}}{\usepackage{hyperref}}
\hypersetup{
  pdftitle={taller 2},
  hidelinks,
  pdfcreator={LaTeX via pandoc}}
\urlstyle{same} % disable monospaced font for URLs
\usepackage[margin=1in]{geometry}
\usepackage{color}
\usepackage{fancyvrb}
\newcommand{\VerbBar}{|}
\newcommand{\VERB}{\Verb[commandchars=\\\{\}]}
\DefineVerbatimEnvironment{Highlighting}{Verbatim}{commandchars=\\\{\}}
% Add ',fontsize=\small' for more characters per line
\usepackage{framed}
\definecolor{shadecolor}{RGB}{248,248,248}
\newenvironment{Shaded}{\begin{snugshade}}{\end{snugshade}}
\newcommand{\AlertTok}[1]{\textcolor[rgb]{0.94,0.16,0.16}{#1}}
\newcommand{\AnnotationTok}[1]{\textcolor[rgb]{0.56,0.35,0.01}{\textbf{\textit{#1}}}}
\newcommand{\AttributeTok}[1]{\textcolor[rgb]{0.77,0.63,0.00}{#1}}
\newcommand{\BaseNTok}[1]{\textcolor[rgb]{0.00,0.00,0.81}{#1}}
\newcommand{\BuiltInTok}[1]{#1}
\newcommand{\CharTok}[1]{\textcolor[rgb]{0.31,0.60,0.02}{#1}}
\newcommand{\CommentTok}[1]{\textcolor[rgb]{0.56,0.35,0.01}{\textit{#1}}}
\newcommand{\CommentVarTok}[1]{\textcolor[rgb]{0.56,0.35,0.01}{\textbf{\textit{#1}}}}
\newcommand{\ConstantTok}[1]{\textcolor[rgb]{0.00,0.00,0.00}{#1}}
\newcommand{\ControlFlowTok}[1]{\textcolor[rgb]{0.13,0.29,0.53}{\textbf{#1}}}
\newcommand{\DataTypeTok}[1]{\textcolor[rgb]{0.13,0.29,0.53}{#1}}
\newcommand{\DecValTok}[1]{\textcolor[rgb]{0.00,0.00,0.81}{#1}}
\newcommand{\DocumentationTok}[1]{\textcolor[rgb]{0.56,0.35,0.01}{\textbf{\textit{#1}}}}
\newcommand{\ErrorTok}[1]{\textcolor[rgb]{0.64,0.00,0.00}{\textbf{#1}}}
\newcommand{\ExtensionTok}[1]{#1}
\newcommand{\FloatTok}[1]{\textcolor[rgb]{0.00,0.00,0.81}{#1}}
\newcommand{\FunctionTok}[1]{\textcolor[rgb]{0.00,0.00,0.00}{#1}}
\newcommand{\ImportTok}[1]{#1}
\newcommand{\InformationTok}[1]{\textcolor[rgb]{0.56,0.35,0.01}{\textbf{\textit{#1}}}}
\newcommand{\KeywordTok}[1]{\textcolor[rgb]{0.13,0.29,0.53}{\textbf{#1}}}
\newcommand{\NormalTok}[1]{#1}
\newcommand{\OperatorTok}[1]{\textcolor[rgb]{0.81,0.36,0.00}{\textbf{#1}}}
\newcommand{\OtherTok}[1]{\textcolor[rgb]{0.56,0.35,0.01}{#1}}
\newcommand{\PreprocessorTok}[1]{\textcolor[rgb]{0.56,0.35,0.01}{\textit{#1}}}
\newcommand{\RegionMarkerTok}[1]{#1}
\newcommand{\SpecialCharTok}[1]{\textcolor[rgb]{0.00,0.00,0.00}{#1}}
\newcommand{\SpecialStringTok}[1]{\textcolor[rgb]{0.31,0.60,0.02}{#1}}
\newcommand{\StringTok}[1]{\textcolor[rgb]{0.31,0.60,0.02}{#1}}
\newcommand{\VariableTok}[1]{\textcolor[rgb]{0.00,0.00,0.00}{#1}}
\newcommand{\VerbatimStringTok}[1]{\textcolor[rgb]{0.31,0.60,0.02}{#1}}
\newcommand{\WarningTok}[1]{\textcolor[rgb]{0.56,0.35,0.01}{\textbf{\textit{#1}}}}
\usepackage{graphicx}
\makeatletter
\def\maxwidth{\ifdim\Gin@nat@width>\linewidth\linewidth\else\Gin@nat@width\fi}
\def\maxheight{\ifdim\Gin@nat@height>\textheight\textheight\else\Gin@nat@height\fi}
\makeatother
% Scale images if necessary, so that they will not overflow the page
% margins by default, and it is still possible to overwrite the defaults
% using explicit options in \includegraphics[width, height, ...]{}
\setkeys{Gin}{width=\maxwidth,height=\maxheight,keepaspectratio}
% Set default figure placement to htbp
\makeatletter
\def\fps@figure{htbp}
\makeatother
\setlength{\emergencystretch}{3em} % prevent overfull lines
\providecommand{\tightlist}{%
  \setlength{\itemsep}{0pt}\setlength{\parskip}{0pt}}
\setcounter{secnumdepth}{-\maxdimen} % remove section numbering
\ifLuaTeX
  \usepackage{selnolig}  % disable illegal ligatures
\fi

\begin{document}
\maketitle

\begin{Shaded}
\begin{Highlighting}[]
\FunctionTok{install.packages}\NormalTok{(}\StringTok{"plotly"}\NormalTok{)}
\end{Highlighting}
\end{Shaded}

\begin{verbatim}
## Installing package into '/home/dave/R/x86_64-pc-linux-gnu-library/3.6'
## (as 'lib' is unspecified)
\end{verbatim}

\begin{verbatim}
## also installing the dependencies 'curl', 'openssl', 'httr'
\end{verbatim}

\begin{verbatim}
## Warning in install.packages("plotly"): installation of package 'curl' had non-
## zero exit status
\end{verbatim}

\begin{verbatim}
## Warning in install.packages("plotly"): installation of package 'openssl' had
## non-zero exit status
\end{verbatim}

\begin{verbatim}
## Warning in install.packages("plotly"): installation of package 'httr' had non-
## zero exit status
\end{verbatim}

\begin{verbatim}
## Warning in install.packages("plotly"): installation of package 'plotly' had non-
## zero exit status
\end{verbatim}

\hypertarget{taller-2}{%
\subsection{Taller 2}\label{taller-2}}

\hypertarget{las-siguientes-son-5-medidas-sobre-las-variables-x_1-x_2-x_3}{%
\paragraph{\texorpdfstring{1. Las siguientes son 5 medidas sobre las
variables
\(x_1, x_2, x_3\):}{1. Las siguientes son 5 medidas sobre las variables x\_1, x\_2, x\_3:}}\label{las-siguientes-son-5-medidas-sobre-las-variables-x_1-x_2-x_3}}

Calculamos las medias de cada variable:

\begin{Shaded}
\begin{Highlighting}[]
\NormalTok{x1 }\OtherTok{=} \FunctionTok{c}\NormalTok{(}\DecValTok{9}\NormalTok{, }\DecValTok{2}\NormalTok{, }\DecValTok{6}\NormalTok{, }\DecValTok{5}\NormalTok{, }\DecValTok{8}\NormalTok{)}
\NormalTok{x2 }\OtherTok{=} \FunctionTok{c}\NormalTok{(}\DecValTok{12}\NormalTok{, }\DecValTok{8}\NormalTok{, }\DecValTok{6}\NormalTok{, }\DecValTok{4}\NormalTok{, }\DecValTok{10}\NormalTok{)}
\NormalTok{x3 }\OtherTok{=} \FunctionTok{c}\NormalTok{(}\DecValTok{3}\NormalTok{, }\DecValTok{4}\NormalTok{, }\DecValTok{0}\NormalTok{, }\DecValTok{2}\NormalTok{, }\DecValTok{1}\NormalTok{)}

\NormalTok{df }\OtherTok{=} \FunctionTok{data.frame}\NormalTok{(x1, x2, x3)}
\NormalTok{mean }\OtherTok{=} \FunctionTok{colMeans}\NormalTok{(df)}
\NormalTok{mean}
\end{Highlighting}
\end{Shaded}

\begin{verbatim}
## x1 x2 x3 
##  6  8  2
\end{verbatim}

Calculamos la matriz \(S_n\) de los datos:

\begin{Shaded}
\begin{Highlighting}[]
\NormalTok{s }\OtherTok{=} \FunctionTok{cov}\NormalTok{(df)}
\NormalTok{s}
\end{Highlighting}
\end{Shaded}

\begin{verbatim}
##       x1   x2    x3
## x1  7.50  5.0 -1.75
## x2  5.00 10.0  1.50
## x3 -1.75  1.5  2.50
\end{verbatim}

Obtenemos la matriz \(R\):

\begin{Shaded}
\begin{Highlighting}[]
\NormalTok{r }\OtherTok{=} \FunctionTok{cor}\NormalTok{(df)}
\NormalTok{r}
\end{Highlighting}
\end{Shaded}

\begin{verbatim}
##            x1        x2         x3
## x1  1.0000000 0.5773503 -0.4041452
## x2  0.5773503 1.0000000  0.3000000
## x3 -0.4041452 0.3000000  1.0000000
\end{verbatim}

\hypertarget{sea-x-5-1-3-y-y--1-3-1}{%
\paragraph{\texorpdfstring{2. Sea \(x' = [5, 1, 3]\) y
\(y = [-1, 3 1]\)}{2. Sea x' = {[}5, 1, 3{]} y y = {[}-1, 3 1{]}}}\label{sea-x-5-1-3-y-y--1-3-1}}

\textbf{A) Grafique los vectores.}

\begin{Shaded}
\begin{Highlighting}[]
\NormalTok{x }\OtherTok{=} \FunctionTok{c}\NormalTok{(}\DecValTok{5}\NormalTok{, }\DecValTok{1}\NormalTok{, }\DecValTok{3}\NormalTok{)}
\NormalTok{y }\OtherTok{=} \FunctionTok{c}\NormalTok{(}\SpecialCharTok{{-}}\DecValTok{1}\NormalTok{, }\DecValTok{3}\NormalTok{, }\DecValTok{1}\NormalTok{)}

\FunctionTok{scatterplot3d}\NormalTok{( }\AttributeTok{x =} \FunctionTok{c}\NormalTok{(x[}\DecValTok{1}\NormalTok{], y[}\DecValTok{1}\NormalTok{]), }\AttributeTok{y =} \FunctionTok{c}\NormalTok{(x[}\DecValTok{2}\NormalTok{], y[}\DecValTok{2}\NormalTok{]), }\AttributeTok{z =} \FunctionTok{c}\NormalTok{(x[}\DecValTok{3}\NormalTok{], y[}\DecValTok{3}\NormalTok{]))}
\end{Highlighting}
\end{Shaded}

\includegraphics{taller2_files/figure-latex/unnamed-chunk-5-1.pdf}

\textbf{B) Encuentre: }

\begin{itemize}
\tightlist
\item
  Longitud de x:
\end{itemize}

\begin{Shaded}
\begin{Highlighting}[]
\NormalTok{x\_lenght }\OtherTok{=} \FunctionTok{sqrt}\NormalTok{(x}\SpecialCharTok{\%*\%}\NormalTok{x)}
\NormalTok{x\_lenght}
\end{Highlighting}
\end{Shaded}

\begin{verbatim}
##         [,1]
## [1,] 5.91608
\end{verbatim}

\begin{itemize}
\tightlist
\item
  Ángulo entre \(x\) e \(y\), recordemos
  \(cos(\theta) = \frac{x \cdot y}{||x|| \cdot ||y||\)
\end{itemize}

\end{document}
