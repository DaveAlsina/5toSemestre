% Options for packages loaded elsewhere
\PassOptionsToPackage{unicode}{hyperref}
\PassOptionsToPackage{hyphens}{url}
%
\documentclass[
]{article}
\title{Taller 2}
\author{Maria Fernanda Palacio, David Santiago Alsina, Estefania
Laverde}
\date{}

\usepackage{amsmath,amssymb}
\usepackage{lmodern}
\usepackage{iftex}
\ifPDFTeX
  \usepackage[T1]{fontenc}
  \usepackage[utf8]{inputenc}
  \usepackage{textcomp} % provide euro and other symbols
\else % if luatex or xetex
  \usepackage{unicode-math}
  \defaultfontfeatures{Scale=MatchLowercase}
  \defaultfontfeatures[\rmfamily]{Ligatures=TeX,Scale=1}
\fi
% Use upquote if available, for straight quotes in verbatim environments
\IfFileExists{upquote.sty}{\usepackage{upquote}}{}
\IfFileExists{microtype.sty}{% use microtype if available
  \usepackage[]{microtype}
  \UseMicrotypeSet[protrusion]{basicmath} % disable protrusion for tt fonts
}{}
\makeatletter
\@ifundefined{KOMAClassName}{% if non-KOMA class
  \IfFileExists{parskip.sty}{%
    \usepackage{parskip}
  }{% else
    \setlength{\parindent}{0pt}
    \setlength{\parskip}{6pt plus 2pt minus 1pt}}
}{% if KOMA class
  \KOMAoptions{parskip=half}}
\makeatother
\usepackage{xcolor}
\IfFileExists{xurl.sty}{\usepackage{xurl}}{} % add URL line breaks if available
\IfFileExists{bookmark.sty}{\usepackage{bookmark}}{\usepackage{hyperref}}
\hypersetup{
  pdftitle={Taller 2},
  pdfauthor={Maria Fernanda Palacio, David Santiago Alsina, Estefania Laverde},
  hidelinks,
  pdfcreator={LaTeX via pandoc}}
\urlstyle{same} % disable monospaced font for URLs
\usepackage[margin=1in]{geometry}
\usepackage{color}
\usepackage{fancyvrb}
\newcommand{\VerbBar}{|}
\newcommand{\VERB}{\Verb[commandchars=\\\{\}]}
\DefineVerbatimEnvironment{Highlighting}{Verbatim}{commandchars=\\\{\}}
% Add ',fontsize=\small' for more characters per line
\usepackage{framed}
\definecolor{shadecolor}{RGB}{248,248,248}
\newenvironment{Shaded}{\begin{snugshade}}{\end{snugshade}}
\newcommand{\AlertTok}[1]{\textcolor[rgb]{0.94,0.16,0.16}{#1}}
\newcommand{\AnnotationTok}[1]{\textcolor[rgb]{0.56,0.35,0.01}{\textbf{\textit{#1}}}}
\newcommand{\AttributeTok}[1]{\textcolor[rgb]{0.77,0.63,0.00}{#1}}
\newcommand{\BaseNTok}[1]{\textcolor[rgb]{0.00,0.00,0.81}{#1}}
\newcommand{\BuiltInTok}[1]{#1}
\newcommand{\CharTok}[1]{\textcolor[rgb]{0.31,0.60,0.02}{#1}}
\newcommand{\CommentTok}[1]{\textcolor[rgb]{0.56,0.35,0.01}{\textit{#1}}}
\newcommand{\CommentVarTok}[1]{\textcolor[rgb]{0.56,0.35,0.01}{\textbf{\textit{#1}}}}
\newcommand{\ConstantTok}[1]{\textcolor[rgb]{0.00,0.00,0.00}{#1}}
\newcommand{\ControlFlowTok}[1]{\textcolor[rgb]{0.13,0.29,0.53}{\textbf{#1}}}
\newcommand{\DataTypeTok}[1]{\textcolor[rgb]{0.13,0.29,0.53}{#1}}
\newcommand{\DecValTok}[1]{\textcolor[rgb]{0.00,0.00,0.81}{#1}}
\newcommand{\DocumentationTok}[1]{\textcolor[rgb]{0.56,0.35,0.01}{\textbf{\textit{#1}}}}
\newcommand{\ErrorTok}[1]{\textcolor[rgb]{0.64,0.00,0.00}{\textbf{#1}}}
\newcommand{\ExtensionTok}[1]{#1}
\newcommand{\FloatTok}[1]{\textcolor[rgb]{0.00,0.00,0.81}{#1}}
\newcommand{\FunctionTok}[1]{\textcolor[rgb]{0.00,0.00,0.00}{#1}}
\newcommand{\ImportTok}[1]{#1}
\newcommand{\InformationTok}[1]{\textcolor[rgb]{0.56,0.35,0.01}{\textbf{\textit{#1}}}}
\newcommand{\KeywordTok}[1]{\textcolor[rgb]{0.13,0.29,0.53}{\textbf{#1}}}
\newcommand{\NormalTok}[1]{#1}
\newcommand{\OperatorTok}[1]{\textcolor[rgb]{0.81,0.36,0.00}{\textbf{#1}}}
\newcommand{\OtherTok}[1]{\textcolor[rgb]{0.56,0.35,0.01}{#1}}
\newcommand{\PreprocessorTok}[1]{\textcolor[rgb]{0.56,0.35,0.01}{\textit{#1}}}
\newcommand{\RegionMarkerTok}[1]{#1}
\newcommand{\SpecialCharTok}[1]{\textcolor[rgb]{0.00,0.00,0.00}{#1}}
\newcommand{\SpecialStringTok}[1]{\textcolor[rgb]{0.31,0.60,0.02}{#1}}
\newcommand{\StringTok}[1]{\textcolor[rgb]{0.31,0.60,0.02}{#1}}
\newcommand{\VariableTok}[1]{\textcolor[rgb]{0.00,0.00,0.00}{#1}}
\newcommand{\VerbatimStringTok}[1]{\textcolor[rgb]{0.31,0.60,0.02}{#1}}
\newcommand{\WarningTok}[1]{\textcolor[rgb]{0.56,0.35,0.01}{\textbf{\textit{#1}}}}
\usepackage{graphicx}
\makeatletter
\def\maxwidth{\ifdim\Gin@nat@width>\linewidth\linewidth\else\Gin@nat@width\fi}
\def\maxheight{\ifdim\Gin@nat@height>\textheight\textheight\else\Gin@nat@height\fi}
\makeatother
% Scale images if necessary, so that they will not overflow the page
% margins by default, and it is still possible to overwrite the defaults
% using explicit options in \includegraphics[width, height, ...]{}
\setkeys{Gin}{width=\maxwidth,height=\maxheight,keepaspectratio}
% Set default figure placement to htbp
\makeatletter
\def\fps@figure{htbp}
\makeatother
\setlength{\emergencystretch}{3em} % prevent overfull lines
\providecommand{\tightlist}{%
  \setlength{\itemsep}{0pt}\setlength{\parskip}{0pt}}
\setcounter{secnumdepth}{-\maxdimen} % remove section numbering
\ifLuaTeX
  \usepackage{selnolig}  % disable illegal ligatures
\fi

\begin{document}
\maketitle

\hypertarget{las-siguientes-son-5-medidas-sobre-las-variables-x_1-x_2-x_3}{%
\paragraph{\texorpdfstring{1. Las siguientes son 5 medidas sobre las
variables
\(x_1, x_2, x_3\):}{1. Las siguientes son 5 medidas sobre las variables x\_1, x\_2, x\_3:}}\label{las-siguientes-son-5-medidas-sobre-las-variables-x_1-x_2-x_3}}

Calculamos las medias de cada variable:

\begin{Shaded}
\begin{Highlighting}[]
\NormalTok{x1 }\OtherTok{=} \FunctionTok{c}\NormalTok{(}\DecValTok{9}\NormalTok{, }\DecValTok{2}\NormalTok{, }\DecValTok{6}\NormalTok{, }\DecValTok{5}\NormalTok{, }\DecValTok{8}\NormalTok{)}
\NormalTok{x2 }\OtherTok{=} \FunctionTok{c}\NormalTok{(}\DecValTok{12}\NormalTok{, }\DecValTok{8}\NormalTok{, }\DecValTok{6}\NormalTok{, }\DecValTok{4}\NormalTok{, }\DecValTok{10}\NormalTok{)}
\NormalTok{x3 }\OtherTok{=} \FunctionTok{c}\NormalTok{(}\DecValTok{3}\NormalTok{, }\DecValTok{4}\NormalTok{, }\DecValTok{0}\NormalTok{, }\DecValTok{2}\NormalTok{, }\DecValTok{1}\NormalTok{)}

\NormalTok{df }\OtherTok{=} \FunctionTok{data.frame}\NormalTok{(x1, x2, x3)}
\NormalTok{mean }\OtherTok{=} \FunctionTok{colMeans}\NormalTok{(df)}
\NormalTok{mean}
\end{Highlighting}
\end{Shaded}

\begin{verbatim}
## x1 x2 x3 
##  6  8  2
\end{verbatim}

Calculamos la matriz \(S_n\) de los datos:

\begin{Shaded}
\begin{Highlighting}[]
\NormalTok{s }\OtherTok{=} \FunctionTok{cov}\NormalTok{(df)}
\NormalTok{s}
\end{Highlighting}
\end{Shaded}

\begin{verbatim}
##       x1   x2    x3
## x1  7.50  5.0 -1.75
## x2  5.00 10.0  1.50
## x3 -1.75  1.5  2.50
\end{verbatim}

Obtenemos la matriz \(R\):

\begin{Shaded}
\begin{Highlighting}[]
\NormalTok{r }\OtherTok{=} \FunctionTok{cor}\NormalTok{(df)}
\NormalTok{r}
\end{Highlighting}
\end{Shaded}

\begin{verbatim}
##            x1        x2         x3
## x1  1.0000000 0.5773503 -0.4041452
## x2  0.5773503 1.0000000  0.3000000
## x3 -0.4041452 0.3000000  1.0000000
\end{verbatim}

\hypertarget{sea-x-5-1-3-y-y--1-3-1}{%
\paragraph{\texorpdfstring{2. Sea \(x' = [5, 1, 3]\) y
\(y = [-1, 3 1]\)}{2. Sea x' = {[}5, 1, 3{]} y y = {[}-1, 3 1{]}}}\label{sea-x-5-1-3-y-y--1-3-1}}

\textbf{A) Grafique los vectores.}

\begin{Shaded}
\begin{Highlighting}[]
\NormalTok{x }\OtherTok{=} \FunctionTok{c}\NormalTok{(}\DecValTok{5}\NormalTok{, }\DecValTok{1}\NormalTok{, }\DecValTok{3}\NormalTok{)}
\NormalTok{y }\OtherTok{=} \FunctionTok{c}\NormalTok{(}\SpecialCharTok{{-}}\DecValTok{1}\NormalTok{, }\DecValTok{3}\NormalTok{, }\DecValTok{1}\NormalTok{)}

\FunctionTok{scatterplot3d}\NormalTok{( }\AttributeTok{x =} \FunctionTok{c}\NormalTok{(x[}\DecValTok{1}\NormalTok{], y[}\DecValTok{1}\NormalTok{]), }\AttributeTok{y =} \FunctionTok{c}\NormalTok{(x[}\DecValTok{2}\NormalTok{], y[}\DecValTok{2}\NormalTok{]), }\AttributeTok{z =} \FunctionTok{c}\NormalTok{(x[}\DecValTok{3}\NormalTok{], y[}\DecValTok{3}\NormalTok{]))}
\end{Highlighting}
\end{Shaded}

\includegraphics{taller2_files/figure-latex/unnamed-chunk-5-1.pdf}

\textbf{B) Encuentre: }

\begin{itemize}
\tightlist
\item
  Longitud de x:
\end{itemize}

\begin{Shaded}
\begin{Highlighting}[]
\NormalTok{x\_lenght }\OtherTok{=} \FunctionTok{sqrt}\NormalTok{(x}\SpecialCharTok{\%*\%}\NormalTok{x)}
\NormalTok{x\_lenght}
\end{Highlighting}
\end{Shaded}

\begin{verbatim}
##         [,1]
## [1,] 5.91608
\end{verbatim}

\begin{itemize}
\tightlist
\item
  Ángulo entre \(x\) e \(y\), recordemos
  \(cos(\theta) = \frac{x \cdot y}{||x|| \cdot ||y||}\)
\end{itemize}

\begin{Shaded}
\begin{Highlighting}[]
\NormalTok{theta }\OtherTok{=} \FunctionTok{acos}\NormalTok{( (x}\SpecialCharTok{\%*\%}\NormalTok{y)}\SpecialCharTok{/}\NormalTok{(}\FunctionTok{sqrt}\NormalTok{(x}\SpecialCharTok{\%*\%}\NormalTok{x)}\SpecialCharTok{*}\FunctionTok{sqrt}\NormalTok{(y}\SpecialCharTok{\%*\%}\NormalTok{y)) )}
\NormalTok{theta }\OtherTok{=}\NormalTok{ theta}\SpecialCharTok{*}\DecValTok{180}\SpecialCharTok{/}\NormalTok{pi}
\NormalTok{theta}
\end{Highlighting}
\end{Shaded}

\begin{verbatim}
##          [,1]
## [1,] 87.07867
\end{verbatim}

\begin{itemize}
\tightlist
\item
  Proyección de \(y\) en \(x\):
\end{itemize}

\begin{Shaded}
\begin{Highlighting}[]
\NormalTok{proyection }\OtherTok{\textless{}{-}} \ControlFlowTok{function}\NormalTok{(a, b)\{}
  \CommentTok{\#proyecta b en a}
\NormalTok{  p }\OtherTok{=}\NormalTok{ ((b}\SpecialCharTok{\%*\%}\NormalTok{a)}\SpecialCharTok{/}\NormalTok{(a}\SpecialCharTok{\%*\%}\NormalTok{a))}\SpecialCharTok{*}\NormalTok{a}
  \FunctionTok{return}\NormalTok{(p)}
\NormalTok{\}}

\FunctionTok{proyection}\NormalTok{(x, y)}
\end{Highlighting}
\end{Shaded}

\begin{verbatim}
## [1] 0.14285714 0.02857143 0.08571429
\end{verbatim}

\textbf{C) Dado que \(\bar{x} = 3\), y \(\bar{y} = 1\), grafique
\(x-\bar{x}\) y \(y-\bar{y}\):}

\begin{Shaded}
\begin{Highlighting}[]
\NormalTok{x }\OtherTok{=}\NormalTok{ x }\SpecialCharTok{{-}} \DecValTok{3}
\NormalTok{y }\OtherTok{=}\NormalTok{ y }\SpecialCharTok{{-}} \DecValTok{1}

\FunctionTok{scatterplot3d}\NormalTok{( }\AttributeTok{x =} \FunctionTok{c}\NormalTok{(x[}\DecValTok{1}\NormalTok{], y[}\DecValTok{1}\NormalTok{]), }\AttributeTok{y =} \FunctionTok{c}\NormalTok{(x[}\DecValTok{2}\NormalTok{], y[}\DecValTok{2}\NormalTok{]), }\AttributeTok{z =} \FunctionTok{c}\NormalTok{(x[}\DecValTok{3}\NormalTok{], y[}\DecValTok{3}\NormalTok{]))}
\end{Highlighting}
\end{Shaded}

\includegraphics{taller2_files/figure-latex/unnamed-chunk-9-1.pdf}

\hypertarget{sea-la-matriz-a}{%
\paragraph{3. Sea la matriz A:}\label{sea-la-matriz-a}}

\begin{Shaded}
\begin{Highlighting}[]
\NormalTok{A }\OtherTok{=} \FunctionTok{cbind}\NormalTok{(}\FunctionTok{c}\NormalTok{(}\DecValTok{9}\NormalTok{, }\SpecialCharTok{{-}}\DecValTok{2}\NormalTok{), }\FunctionTok{c}\NormalTok{(}\SpecialCharTok{{-}}\DecValTok{2}\NormalTok{, }\DecValTok{6}\NormalTok{))}
\NormalTok{A}
\end{Highlighting}
\end{Shaded}

\begin{verbatim}
##      [,1] [,2]
## [1,]    9   -2
## [2,]   -2    6
\end{verbatim}

\begin{itemize}
\tightlist
\item
  ¿Es simétrica? \textbf{(si es simétrica, pues su transpuesta es igual
  a la matriz)}
\end{itemize}

\begin{Shaded}
\begin{Highlighting}[]
\FunctionTok{t}\NormalTok{(A) }\SpecialCharTok{==}\NormalTok{ A}
\end{Highlighting}
\end{Shaded}

\begin{verbatim}
##      [,1] [,2]
## [1,] TRUE TRUE
## [2,] TRUE TRUE
\end{verbatim}

\begin{itemize}
\tightlist
\item
  Muestre que \emph{A} es definida positiva, \emph{(dado que todos los
  eigenvalues son positivos A es definida positiva)}:
\end{itemize}

\begin{Shaded}
\begin{Highlighting}[]
\NormalTok{ans }\OtherTok{=} \FunctionTok{eigen}\NormalTok{(A)}
\NormalTok{ans}\SpecialCharTok{$}\NormalTok{values}
\end{Highlighting}
\end{Shaded}

\begin{verbatim}
## [1] 10  5
\end{verbatim}

Para calcular los eigenvectores con cada eigenvalor, se desarrolla el
siguiente procedimiento: - \(\lambda =10\):

\[
\left(\begin{array}{cc} 
-1 & -2\\
-2 & -4
\end{array}\right)
\left(\begin{array}{cc} 
x\\ 
y
\end{array}\right) = 
\left(\begin{array}{cc} 
0\\ 
0
\end{array}\right)
\] Luego se tiene el sistema de ecuaciones:

\[
-x-2y = 0
\] \[
-2x-4y = 0
\] Y al resolverlo se tiene entonces que \[
V1 = \left(\begin{array}{cc} 
-2a\\
a
\end{array}\right), \forall a \in \mathbf{R}
\] Si tomamos al valor \(a\) como 1, y lo dividimos entre su módulo
(\(\sqrt{5}\)), se obtiene el eigenvector conseguido con el código.

El mismo procedimeinto se realiza con \(\lambda = 5\), y se obtiene el
eigenvector

\[
V1 = \left(\begin{array}{cc} 
a\\
2a
\end{array}\right), \forall a \in \mathbf{R}
\] Al tomar \(a\) como 1, y dividiéndolo entre su módulo (\(\sqrt{5}\)),
se obtiene el vector del código ensenado a continuación.

\begin{Shaded}
\begin{Highlighting}[]
\NormalTok{ans}\SpecialCharTok{$}\NormalTok{vectors}
\end{Highlighting}
\end{Shaded}

\begin{verbatim}
##            [,1]       [,2]
## [1,] -0.8944272 -0.4472136
## [2,]  0.4472136 -0.8944272
\end{verbatim}

\begin{itemize}
\tightlist
\item
  La matriz de descomposición espectral es:
\end{itemize}

\begin{Shaded}
\begin{Highlighting}[]
\NormalTok{A1 }\OtherTok{=}\NormalTok{ ans}\SpecialCharTok{$}\NormalTok{values[}\DecValTok{1}\NormalTok{]}\SpecialCharTok{*}\NormalTok{ans}\SpecialCharTok{$}\NormalTok{vectors[,}\DecValTok{1}\NormalTok{] }\SpecialCharTok{\%*\%} \FunctionTok{t}\NormalTok{(ans}\SpecialCharTok{$}\NormalTok{vector[,}\DecValTok{1}\NormalTok{])}
\NormalTok{A1}
\end{Highlighting}
\end{Shaded}

\begin{verbatim}
##      [,1] [,2]
## [1,]    8   -4
## [2,]   -4    2
\end{verbatim}

\begin{Shaded}
\begin{Highlighting}[]
\NormalTok{A2 }\OtherTok{=}\NormalTok{ ans}\SpecialCharTok{$}\NormalTok{values[}\DecValTok{2}\NormalTok{]}\SpecialCharTok{*}\NormalTok{ans}\SpecialCharTok{$}\NormalTok{vectors[,}\DecValTok{2}\NormalTok{] }\SpecialCharTok{\%*\%} \FunctionTok{t}\NormalTok{(ans}\SpecialCharTok{$}\NormalTok{vector[,}\DecValTok{2}\NormalTok{])}
\NormalTok{A2}
\end{Highlighting}
\end{Shaded}

\begin{verbatim}
##      [,1] [,2]
## [1,]    1    2
## [2,]    2    4
\end{verbatim}

\begin{Shaded}
\begin{Highlighting}[]
\NormalTok{desc }\OtherTok{=}\NormalTok{ A1}\SpecialCharTok{+}\NormalTok{A2}
\NormalTok{desc}
\end{Highlighting}
\end{Shaded}

\begin{verbatim}
##      [,1] [,2]
## [1,]    9   -2
## [2,]   -2    6
\end{verbatim}

Como A1 y A2 forman A, estas son las componentes de la descomposición
espectral.

\begin{itemize}
\tightlist
\item
  La inversa de A es:
\end{itemize}

\begin{Shaded}
\begin{Highlighting}[]
\NormalTok{A\_inv }\OtherTok{=} \FunctionTok{solve}\NormalTok{(A)}
\NormalTok{A\_inv}
\end{Highlighting}
\end{Shaded}

\begin{verbatim}
##      [,1] [,2]
## [1,] 0.12 0.04
## [2,] 0.04 0.18
\end{verbatim}

\begin{itemize}
\tightlist
\item
  Encuentre los valores y vectores propios de \(A^{-1}\):
\end{itemize}

\begin{Shaded}
\begin{Highlighting}[]
\NormalTok{ans2 }\OtherTok{=} \FunctionTok{eigen}\NormalTok{(A\_inv)}
\FunctionTok{print}\NormalTok{(ans2}\SpecialCharTok{$}\NormalTok{values)}
\end{Highlighting}
\end{Shaded}

\begin{verbatim}
## [1] 0.2 0.1
\end{verbatim}

\begin{Shaded}
\begin{Highlighting}[]
\FunctionTok{print}\NormalTok{(ans}\SpecialCharTok{$}\NormalTok{vectors)}
\end{Highlighting}
\end{Shaded}

\begin{verbatim}
##            [,1]       [,2]
## [1,] -0.8944272 -0.4472136
## [2,]  0.4472136 -0.8944272
\end{verbatim}

\hypertarget{verifique-las-relaciones-v12rho-v12-sigma-y-v12-1-sigma-v12-1-rho-donde-sigma-es-la-pxp-matriz-de-covarianza-poblacional-rho-es-la-matriz-de-correlaciuxf3n-poblacional-pxp-y-v12-es-la-natriz-de-desviaciuxf3n-estuxe1ndar-de-la-poblaciuxf3n.}{%
\paragraph{\texorpdfstring{4. Verifique las relaciones
\(V^{1/2}\rho V^{1/2} = \Sigma\) y
\((V^{1/2})^{-1} \Sigma (V^{1/2})^{-1} = \rho\), donde \(\Sigma\) es la
\(pxp\) matriz de covarianza poblacional, \(\rho\) es la matriz de
correlación poblacional pxp y \(V^{1/2}\) es la natriz de desviación
estándar de la
población.}{4. Verifique las relaciones V\^{}\{1/2\}\textbackslash rho V\^{}\{1/2\} = \textbackslash Sigma y (V\^{}\{1/2\})\^{}\{-1\} \textbackslash Sigma (V\^{}\{1/2\})\^{}\{-1\} = \textbackslash rho, donde \textbackslash Sigma es la pxp matriz de covarianza poblacional, \textbackslash rho es la matriz de correlación poblacional pxp y V\^{}\{1/2\} es la natriz de desviación estándar de la población.}}\label{verifique-las-relaciones-v12rho-v12-sigma-y-v12-1-sigma-v12-1-rho-donde-sigma-es-la-pxp-matriz-de-covarianza-poblacional-rho-es-la-matriz-de-correlaciuxf3n-poblacional-pxp-y-v12-es-la-natriz-de-desviaciuxf3n-estuxe1ndar-de-la-poblaciuxf3n.}}

Sea \(V^{1/2}\rho V^{1/2} = \Sigma\), queremos verificar que
\(\rho = (V^{1/2})^{-1} \Sigma (V^{1/2})^{-1}\).

Sustituyendo el valor de \(\Sigma\) en la segunda ecuación se obtiene
\((V^{1/2})^{-1} [V^{1/2}\rho V^{1/2}] (V^{1/2})^{-1}\)

Note que \([(V^{1/2})^{-1} V^{1/2}] = [V^{1/2} (V^{1/2})^{-1}] = I\).

Y por lo tanto,
\((V^{1/2})^{-1} [V^{1/2}\rho V^{1/2}] (V^{1/2})^{-1} = I \rho I = \rho\).

Por otro lado, conociendo que
\(\rho = (V^{1/2})^{-1} \Sigma (V^{1/2})^{-1}\) queremos verificar que
\(V^{1/2}\rho V^{1/2} = \Sigma\).

De la misma manera, sustituimos el valor de \(\rho\) en la segunda
ecuación de modo que
\(V^{1/2}[(V^{1/2})^{-1} \Sigma (V^{1/2})^{-1}]V^{1/2} = I \Sigma I = \Sigma\)

Concluimos así que las relaciones se cumplen.

\hypertarget{derive-las-expresiones-para-la-media-y-las-varianzas-de-las-siguientes-combinaciones-lineales-en-terminos-de-lsas-meduas-y-covaruanzas-de-las-variables-aleatorias-x_1-x_2-y-x_3.}{%
\paragraph{\texorpdfstring{5. Derive las expresiones para la media y las
varianzas de las siguientes combinaciones lineales en terminos de ls¿as
meduas y covaruanzas de las variables aleatorias \(X_1\), \(X_2\) y
\(X_3\).}{5. Derive las expresiones para la media y las varianzas de las siguientes combinaciones lineales en terminos de ls¿as meduas y covaruanzas de las variables aleatorias X\_1, X\_2 y X\_3.}}\label{derive-las-expresiones-para-la-media-y-las-varianzas-de-las-siguientes-combinaciones-lineales-en-terminos-de-lsas-meduas-y-covaruanzas-de-las-variables-aleatorias-x_1-x_2-y-x_3.}}

\textbf{A) \(X_1-2X_2\)}

\(E(X_1-2X_2) = E(X_1)-2E(X_2)\)

\(var(X_1-2X_2) = var(X_1)+4var(X_2)+4cov(X_1,X_2)\)

\textbf{B) \(-X_1+3X_2\)}

\(E(-X_1+3X_2) = -E(X_1)+3E(X_2)\)

\(var(-X_1+3X_2) = var(-X_1)+9var(X_2)+2cov(-X_1,3X_2) = var(X_1)+9var(X_2)-6cov(X_1,X_2)\)

\textbf{C) \(X_1+X_2+X_3\)}

\(E(X_1+X_2+X_3) = E(X_1)+E(X_2)+E(X_3)\)

\(var(X_1+X_2+X_3) = var(X_1)+var(X_2+X_3)+2cov(X_1,X_2+X_3) =\)

\(=var(X_1)+var(X_2)+var(X_3)+2cov(X_2,X_3)+2cov(X_1,X_2)+2cov(X_1,X_3)\)

\textbf{D) \(X_1+2X_2-X_3\)}

\(E(X_1+2X_2-X_3) = E(X_1)+2E(X_2)-E(X_3)\)

\(var(X_1+2X_2-X_3)= var(X_1)+var(2X_2-X_3)+2cov(X_1,2X_2-X_3) =\)

\(var(X_1)+4var(X_2)+var(X_3)-4cov(X_2,X_3)+4cov(X_1,X_2)-2cov(X_1,X_3)\)

\textbf{E) \(3X_1-4X_2\) si \(X_1\) y \(X_2\) son independientes. }

\(E(3X_1-4X_2) = 3E(X_1)-4E(X_2)\)

\(var(3X_1-4X_2)=9var(X_1)+16var(X_2)\)

\hypertarget{dada-la-matriz-de-datos}{%
\paragraph{6. Dada la matriz de datos}\label{dada-la-matriz-de-datos}}

\begin{Shaded}
\begin{Highlighting}[]
\NormalTok{a }\OtherTok{=} \FunctionTok{c}\NormalTok{(}\DecValTok{9}\NormalTok{,}\DecValTok{5}\NormalTok{,}\DecValTok{1}\NormalTok{)}
\NormalTok{b }\OtherTok{=} \FunctionTok{c}\NormalTok{(}\DecValTok{1}\NormalTok{, }\DecValTok{3}\NormalTok{, }\DecValTok{2}\NormalTok{)}

\NormalTok{X }\OtherTok{=} \FunctionTok{cbind}\NormalTok{(a, b)}
\NormalTok{X}
\end{Highlighting}
\end{Shaded}

\begin{verbatim}
##      a b
## [1,] 9 1
## [2,] 5 3
## [3,] 1 2
\end{verbatim}

\textbf{A) Grafique el diagrama de dispersion en \(p=2\) dimensiones.
Localice la media de la muestra en su diagrama}

\begin{Shaded}
\begin{Highlighting}[]
\FunctionTok{plot}\NormalTok{(X, }\AttributeTok{xlim=}\FunctionTok{range}\NormalTok{(}\DecValTok{0}\NormalTok{,}\DecValTok{10}\NormalTok{), }\AttributeTok{pch=}\DecValTok{19}\NormalTok{, }\AttributeTok{col=}\StringTok{"red"}\NormalTok{,}
     \AttributeTok{ylim=}\FunctionTok{range}\NormalTok{(}\DecValTok{0}\NormalTok{, }\DecValTok{5}\NormalTok{), }\AttributeTok{xlab=}\StringTok{\textquotesingle{}eje x\textquotesingle{}}\NormalTok{, }\AttributeTok{ylab=}\StringTok{\textquotesingle{}eje y\textquotesingle{}}\NormalTok{)}
\FunctionTok{par}\NormalTok{(}\AttributeTok{new=}\NormalTok{T)}
\FunctionTok{plot}\NormalTok{(}\FunctionTok{mean}\NormalTok{(a), }\FunctionTok{mean}\NormalTok{(b), }\AttributeTok{xlim=}\FunctionTok{range}\NormalTok{(}\DecValTok{0}\NormalTok{,}\DecValTok{10}\NormalTok{), }\AttributeTok{ylim=}\FunctionTok{range}\NormalTok{(}\DecValTok{0}\NormalTok{,}\DecValTok{5}\NormalTok{), }\AttributeTok{xlab=}\StringTok{\textquotesingle{}eje x\textquotesingle{}}\NormalTok{, }\AttributeTok{ylab=}\StringTok{\textquotesingle{}eje y\textquotesingle{}}\NormalTok{)}
\end{Highlighting}
\end{Shaded}

\includegraphics{taller2_files/figure-latex/unnamed-chunk-20-1.pdf}

\textbf{B) Dibuje la representación \(n=3\) -dimensional de los datos y
trace vectores de desviación \(y_{1} - \bar{x}_{1} \cdot 1\)}

\begin{Shaded}
\begin{Highlighting}[]
\NormalTok{X }\OtherTok{=} \FunctionTok{cbind}\NormalTok{(a, b)}

\FunctionTok{scatterplot3d}\NormalTok{(}\AttributeTok{x =}\NormalTok{ X[}\DecValTok{1}\NormalTok{,], }\AttributeTok{y =}\NormalTok{ X[}\DecValTok{2}\NormalTok{,], }\AttributeTok{z =}\NormalTok{ X[}\DecValTok{3}\NormalTok{,], }\AttributeTok{xlab =} \StringTok{"eje x"}\NormalTok{, }\AttributeTok{ylab =} \StringTok{"eje y"}\NormalTok{,}
              \AttributeTok{zlab =} \StringTok{"eje z"}\NormalTok{, }\AttributeTok{pch =} \DecValTok{16}\NormalTok{, }\AttributeTok{color =} \StringTok{\textquotesingle{}red\textquotesingle{}}\NormalTok{,}
              \AttributeTok{xlim=}\FunctionTok{c}\NormalTok{(}\SpecialCharTok{{-}}\DecValTok{5}\NormalTok{,}\DecValTok{10}\NormalTok{), }\AttributeTok{ylim=}\FunctionTok{c}\NormalTok{(}\SpecialCharTok{{-}}\DecValTok{5}\NormalTok{,}\DecValTok{10}\NormalTok{), }\AttributeTok{zlim =}\FunctionTok{c}\NormalTok{(}\SpecialCharTok{{-}}\DecValTok{5}\NormalTok{,}\DecValTok{10}\NormalTok{))}
\FunctionTok{par}\NormalTok{(}\AttributeTok{new=}\NormalTok{T)}

\NormalTok{x1\_mean }\OtherTok{=} \FunctionTok{mean}\NormalTok{(X[,}\DecValTok{1}\NormalTok{])}
\NormalTok{x2\_mean }\OtherTok{=} \FunctionTok{mean}\NormalTok{(X[,}\DecValTok{2}\NormalTok{])}
\NormalTok{X[,}\DecValTok{1}\NormalTok{] }\OtherTok{=}\NormalTok{ X[,}\DecValTok{1}\NormalTok{] }\SpecialCharTok{{-}}\NormalTok{ x1\_mean}
\NormalTok{X[,}\DecValTok{2}\NormalTok{] }\OtherTok{=}\NormalTok{ X[,}\DecValTok{2}\NormalTok{] }\SpecialCharTok{{-}}\NormalTok{ x2\_mean}

\FunctionTok{scatterplot3d}\NormalTok{(}\AttributeTok{x =}\NormalTok{ X[}\DecValTok{1}\NormalTok{,], }\AttributeTok{y =}\NormalTok{ X[}\DecValTok{2}\NormalTok{,], }\AttributeTok{z =}\NormalTok{ X[}\DecValTok{3}\NormalTok{,], }\AttributeTok{xlab =} \StringTok{"eje x"}\NormalTok{, }\AttributeTok{ylab =} \StringTok{"eje y"}\NormalTok{,}
              \AttributeTok{zlab =} \StringTok{"eje z"}\NormalTok{, }\AttributeTok{pch =} \DecValTok{16}\NormalTok{, }\AttributeTok{color =} \StringTok{\textquotesingle{}blue\textquotesingle{}}\NormalTok{, }
              \AttributeTok{xlim=}\FunctionTok{c}\NormalTok{(}\SpecialCharTok{{-}}\DecValTok{5}\NormalTok{,}\DecValTok{10}\NormalTok{), }\AttributeTok{ylim=}\FunctionTok{c}\NormalTok{(}\SpecialCharTok{{-}}\DecValTok{5}\NormalTok{,}\DecValTok{10}\NormalTok{), }\AttributeTok{zlim =} \FunctionTok{c}\NormalTok{(}\SpecialCharTok{{-}}\DecValTok{5}\NormalTok{, }\DecValTok{10}\NormalTok{))}
\end{Highlighting}
\end{Shaded}

\includegraphics{taller2_files/figure-latex/unnamed-chunk-21-1.pdf}

\textbf{C) Dibuje los vectores de desviación en \((B)\) que emanan del
origen. Calcula las longitudes de estos vectores y del coseno del ángulo
entre ellos. Relacione estas cantidades con \(S_{n}\) y \(R\).}

\begin{Shaded}
\begin{Highlighting}[]
\FunctionTok{scatterplot3d}\NormalTok{(}\AttributeTok{x =}\NormalTok{ X[}\DecValTok{1}\NormalTok{,], }\AttributeTok{y =}\NormalTok{ X[}\DecValTok{2}\NormalTok{,], }\AttributeTok{z =}\NormalTok{ X[}\DecValTok{3}\NormalTok{,], }\AttributeTok{xlab =} \StringTok{"eje x"}\NormalTok{, }\AttributeTok{ylab =} \StringTok{"eje y"}\NormalTok{,}
              \AttributeTok{zlab =} \StringTok{"eje z"}\NormalTok{, }\AttributeTok{pch =} \DecValTok{16}\NormalTok{, }\AttributeTok{color =} \StringTok{\textquotesingle{}blue\textquotesingle{}}\NormalTok{, }
              \AttributeTok{xlim=}\FunctionTok{c}\NormalTok{(}\SpecialCharTok{{-}}\DecValTok{5}\NormalTok{,}\DecValTok{10}\NormalTok{), }\AttributeTok{ylim=}\FunctionTok{c}\NormalTok{(}\SpecialCharTok{{-}}\DecValTok{5}\NormalTok{,}\DecValTok{10}\NormalTok{), }\AttributeTok{zlim =} \FunctionTok{c}\NormalTok{(}\SpecialCharTok{{-}}\DecValTok{5}\NormalTok{, }\DecValTok{10}\NormalTok{))}
\end{Highlighting}
\end{Shaded}

\includegraphics{taller2_files/figure-latex/unnamed-chunk-22-1.pdf}

\begin{Shaded}
\begin{Highlighting}[]
\FunctionTok{print}\NormalTok{(}\FunctionTok{paste}\NormalTok{(}\StringTok{"longitud del vector x1: "}\NormalTok{, }\FunctionTok{sqrt}\NormalTok{(X[,}\DecValTok{1}\NormalTok{]}\SpecialCharTok{\%*\%}\NormalTok{X[,}\DecValTok{1}\NormalTok{]) ) ) }
\end{Highlighting}
\end{Shaded}

\begin{verbatim}
## [1] "longitud del vector x1:  5.65685424949238"
\end{verbatim}

\begin{Shaded}
\begin{Highlighting}[]
\FunctionTok{print}\NormalTok{(}\FunctionTok{paste}\NormalTok{(}\StringTok{"longitud del vector x2: "}\NormalTok{, }\FunctionTok{sqrt}\NormalTok{(X[,}\DecValTok{2}\NormalTok{]}\SpecialCharTok{\%*\%}\NormalTok{X[,}\DecValTok{2}\NormalTok{]) ) ) }
\end{Highlighting}
\end{Shaded}

\begin{verbatim}
## [1] "longitud del vector x2:  1.4142135623731"
\end{verbatim}

\begin{Shaded}
\begin{Highlighting}[]
\FunctionTok{print}\NormalTok{(}\FunctionTok{paste}\NormalTok{(}\StringTok{"coseno del angulo entre ellos: "}\NormalTok{, }
\NormalTok{            (X[,}\DecValTok{1}\NormalTok{]}\SpecialCharTok{\%*\%}\NormalTok{X[,}\DecValTok{2}\NormalTok{])}\SpecialCharTok{/}\NormalTok{(}\FunctionTok{sqrt}\NormalTok{(X[,}\DecValTok{1}\NormalTok{]}\SpecialCharTok{\%*\%}\NormalTok{X[,}\DecValTok{1}\NormalTok{])}\SpecialCharTok{*}\FunctionTok{sqrt}\NormalTok{(X[,}\DecValTok{2}\NormalTok{]}\SpecialCharTok{\%*\%}\NormalTok{X[,}\DecValTok{2}\NormalTok{]))  ))}
\end{Highlighting}
\end{Shaded}

\begin{verbatim}
## [1] "coseno del angulo entre ellos:  -0.5"
\end{verbatim}

Note que el coseno del ángulo entre los vectores es la correlación
(\(R\)) entre sus cantidades.

\begin{Shaded}
\begin{Highlighting}[]
\FunctionTok{cor}\NormalTok{(X)}
\end{Highlighting}
\end{Shaded}

\begin{verbatim}
##      a    b
## a  1.0 -0.5
## b -0.5  1.0
\end{verbatim}

\hypertarget{demuestre-que-s-s_11s_22...s_ppr}{%
\paragraph{\texorpdfstring{7. Demuestre que
\(|S| = (s_{11}s_{22}...s_{pp})|R|\)}{7. Demuestre que \textbar S\textbar{} = (s\_\{11\}s\_\{22\}...s\_\{pp\})\textbar R\textbar{}}}\label{demuestre-que-s-s_11s_22...s_ppr}}

Queremos mostrar que \(|S| = (S_{11}S_{22}...S_{pp})|R|\). Como
\(R = D^{-\frac{1}{2}}S D^{-\frac{1}{2}}\), donde \(D^{-\frac{1}{2}}\)
es una matriz triangular cuyo determinante esta dado por el producto de
los elementos de su diagonal principal, es decir,
\(\frac{1}{\sqrt{S11}\sqrt{S22}...\sqrt{Spp}}\), podemos escribir el
lado derecho de la igualdad como
\((s_{11}s_{22}...s_{pp})|R| = (s_{11}s_{22}...s_{pp})|D^{-\frac{1}{2}}S D^{-\frac{1}{2}}| = (s_{11}s_{22}...s_{pp})|D^{-\frac{1}{2}}||S|| D^{-\frac{1}{2}}|=\)

\(=(s_{11}s_{22}...s_{pp})\frac{1}{\sqrt{S11}\sqrt{S22}...\sqrt{Spp}}|S|\frac{1}{\sqrt{S11}\sqrt{S22}...\sqrt{Spp}} = (s_{11}s_{22}...s_{pp})\frac{1}{S11S22...Spp}|S|=|S|\)

\hypertarget{sea-v-una-variable-aleatoria-vectorial-con-un-vector-medio-ev-mu_v-y-una-matriz-de-covarianza-ev---mu_vcdot-v---mu_v-sigma_v.-demuestre-que}{%
\paragraph{\texorpdfstring{8. Sea \(V\) una variable aleatoria vectorial
con un vector medio \(E(v) = \mu_{v}\) y una matriz de covarianza
\(E((V - \mu_{v})\cdot (V - \mu_{v})') = \Sigma_{v}\). Demuestre
que:}{8. Sea V una variable aleatoria vectorial con un vector medio E(v) = \textbackslash mu\_\{v\} y una matriz de covarianza E((V - \textbackslash mu\_\{v\})\textbackslash cdot (V - \textbackslash mu\_\{v\})') = \textbackslash Sigma\_\{v\}. Demuestre que:}}\label{sea-v-una-variable-aleatoria-vectorial-con-un-vector-medio-ev-mu_v-y-una-matriz-de-covarianza-ev---mu_vcdot-v---mu_v-sigma_v.-demuestre-que}}

\(E(VV') = \Sigma_{v} + \mu_{v}\mu_{v}'\)

Para esta demostracion nos valdremos de las siguientes propiedades. Sean
\(x\) e \(y\) variables aleatorias: \(var(x) = E(x^2) - E(x)^2\),
\(cov(x, y) = E(xy) - E(x)E(y)\).

Empecemos por analizar la forma de \(VV'\):

\[
VV' = \begin{bmatrix}
            x_{1} \\ x_{2} \\ \vdots \\ x_{n}
      \end{bmatrix}
      \begin{bmatrix}
            x_{1} & x_{2} & \dots & x_{n}
      \end{bmatrix} = 
      \begin{bmatrix}
            x_{1}^{2} & \dots & x_{i + n}x_{1} \\
            \vdots    & \ddots & \vdots \\
            x_{n}x_{1} & \dots & x_{i + n}^{2}
      \end{bmatrix}
\]

\[
E(VV') = E\bigg(\begin{bmatrix}
            x_{1} \\ x_{2} \\ \vdots \\ x_{n}
      \end{bmatrix}
      \begin{bmatrix}
            x_{1} & x_{2} & \dots & x_{n}
      \end{bmatrix}\bigg) = 
      \begin{bmatrix}
            E(x_{1}^{2}) & \dots & E(x_{i + n}x_{1}) \\
            \vdots    & \ddots & \vdots \\
            E(x_{n}x_{1}) & \dots & E(x_{i + n}^{2})
      \end{bmatrix}
\]

Vea que \(\mu_{v}\mu_{v}'\) es otra matriz de la forma:

\[
\mu_{v}\mu_{v}' = \begin{bmatrix}
            \mu_{1} \\ \mu_{2} \\ \vdots \\ \mu_{n}
      \end{bmatrix}
      \begin{bmatrix}
            \mu_{1} & \mu_{2} & \dots & \mu_{n}
      \end{bmatrix} = 
      \begin{bmatrix}
            \mu_{1}^{2} & \dots & \mu_{i + n}\mu_{1} \\
            \vdots    & \ddots & \vdots \\
            \mu_{n}\mu_{1} & \dots & \mu_{i + n}^{2}
      \end{bmatrix}
\]

Tenemos entonces que:

\[
\Sigma_{v} =       
      \begin{bmatrix}
            E(x_{1}^{2}) & \dots & E(x_{i + n}x_{1}) \\
            \vdots    & \ddots & \vdots \\
            E(x_{n}x_{1}) & \dots & E(x_{i + n}^{2})
      \end{bmatrix} - 
      \begin{bmatrix}
            \mu_{1}^{2} & \dots & \mu_{i + n}\mu_{1} \\
            \vdots    & \ddots & \vdots \\
            \mu_{n}\mu_{1} & \dots & \mu_{i + n}^{2}
      \end{bmatrix} 
      = 
      \begin{bmatrix}
            s_{1}^{2} & \dots & s_{i + n}s_{1} \\
            \vdots    & \ddots & \vdots \\
            s_{n}s_{1} & \dots & s_{i + n}^{2}
      \end{bmatrix} 
\] Esto es válido porque sabemos que dado \(x\) e \(y\) variables
aleatorias: \(var(x) = E(x^2) - E(x)^2\),
\(cov(x, y) = E(xy) - E(x)E(y)\).

Reordenando todo:

\[
\Sigma_{v} + 
      \begin{bmatrix}
                  \mu_{1}^{2} & \dots & \mu_{i + n}\mu_{1} \\
                  \vdots    & \ddots & \vdots \\
                  \mu_{n}\mu_{1} & \dots & \mu_{i + n}^{2}
      \end{bmatrix} = \Sigma_{v} +  \mu_{v}\mu_{v}' =      
      \begin{bmatrix}
            E(x_{1}^{2}) & \dots & E(x_{i + n}x_{1}) \\
            \vdots    & \ddots & \vdots \\
            E(x_{n}x_{1}) & \dots & E(x_{i + n}^{2})
      \end{bmatrix} 
      = E(VV') \;\;
      \blacksquare
\]

\hypertarget{considere-una-distribuciuxf3n-normal-bivariada-con-mu_1-1-mu_23-sigma_11-2-sigma_22-1-y-p_12--.8}{%
\paragraph{\texorpdfstring{9) Considere una distribución normal
bivariada con \(\mu_{1} = 1\), \(\mu_{2}=3\), \(\sigma_{11} = 2\),
\(\sigma_{22} =1\) y
\(p_{12} =-.8\)}{9) Considere una distribución normal bivariada con \textbackslash mu\_\{1\} = 1, \textbackslash mu\_\{2\}=3, \textbackslash sigma\_\{11\} = 2, \textbackslash sigma\_\{22\} =1 y p\_\{12\} =-.8}}\label{considere-una-distribuciuxf3n-normal-bivariada-con-mu_1-1-mu_23-sigma_11-2-sigma_22-1-y-p_12--.8}}

\textbf{a)} Escriba la densidad normal bivariada.\\
\(p_{12} = \frac{\sigma_{12}}{\sqrt{\sigma_{11}}\sqrt{\sigma_{22}}}\)\\
\(-.8 = \sigma_{12} \frac{1}{\sqrt{2}\sqrt{1}}\)
\(\sigma_{12} = -.8\sqrt{2}\)

\[
\Sigma= \left(\begin{array}
12 & -\sqrt{2}\cdot0.8\\
-\sqrt{2}\cdot0.8 & 1
\end{array}\right)
\] Necesitamos \(|\Sigma^{-1}|\) que calculada en R da: \[
\Sigma^{-1}= \left(\begin{array}
11.3889 & 1.5713\\
1.5713 & 2.7778
\end{array}\right)
\] Y el determinante de esta matriz: \(det(\Sigma^{-1}) = 1.3889\)\\
Con estos datos entonces la formula es:\\
\(f(x_{1}, x_{2}) = \frac{2}{2\pi \cdot 1.3889} \cdot exp((x_{1}-1)^2 \cdot 1.3889 + 2(x_2 -3)(x_1 -1)1.5713 + (x_2 -3)^22.7778 \cdot \frac{-1}{2})\)

\textbf{b)} Escriba la expresión de distancia estadística al cuadrado
\((x − \mu)′ \Sigma^{-1}(x − \mu)\) como una función cuadrática de
\(x_1\) y \(x_2\).

\[
(distancia)^{2} =\left( \begin{bmatrix}
x_1 \\
x_2 
\end{bmatrix}- \left[\begin{array}
11 \\
3 
\end{array}\right]\right)^{'} \begin{bmatrix}
1.3889 & 1.5713  \\
1.5713 & 2.7778
\end{bmatrix} \left( \begin{bmatrix}
x_{1} \\
x_{2}
\end{bmatrix} - \begin{bmatrix}
1 \\
3
\end{bmatrix}\right)
\]

\[
(distancia)^2 = \begin{bmatrix}
x_{1}-1 & x_{2}-3
\end{bmatrix}  \begin{bmatrix}
1.3889 & 1.5713  \\
1.5713 & 2.7778
\end{bmatrix} \begin{bmatrix}
x_{1} -1 & x_{2}-3
\end{bmatrix}
\]

\[ 
(distancia)^2 = [\begin{array} {cc}
1.3889(x_1 -1) + 1.5713(x_2 -3) &  1.5713(x_1 -1)+2.7778(x_2 -3) \end{array} ]  \left[  \begin{array}{cc}
x_{1} -1 \\
x_{2}-3
\end{array} \right]
\]

\[
(distancia)^2 = 1.3889(x_1 -1)^2 + 1.5713(x_2 -3)(x_1 -1) + 1.5713(x_1 -1)(x_2 -3) + 2.7778(x_2 -3)
\]

\hypertarget{sea-x-n_3mu-sigma-con-mu--314-y}{%
\paragraph{\texorpdfstring{10) Sea \(X\) \(N_{3}(\mu, \Sigma)\) con
\(\mu^{'} = [-3,1,4]\)
y}{10) Sea X N\_\{3\}(\textbackslash mu, \textbackslash Sigma) con \textbackslash mu\^{}\{'\} = {[}-3,1,4{]} y}}\label{sea-x-n_3mu-sigma-con-mu--314-y}}

\[
\left(\begin{array} {cc} 
1 & -2 & 0\\
-2 & 5 & 0 \\
0 & 0 & 2
\end{array}\right)
\]

A)\(X_{1}\) y \(X_{2}\)

\begin{quote}
\(\sigma_{12} = \sigma_{21} = -2\) por lo tanto, son dependientes.
\end{quote}

\begin{enumerate}
\def\labelenumi{\Alph{enumi})}
\setcounter{enumi}{1}
\tightlist
\item
  \(X_{2}\) y \(X_{3}\)
\end{enumerate}

\begin{quote}
\(\sigma_{13} = \sigma_{31} = 0\) por lo tanto, son independientes.
\end{quote}

\end{document}
